

\section{Project Information}
\subsection{Scheme Description}
Sparse matrices are extensively used in scientific computing, however there is no automatic differentiation package in Julia yet to handle sparse matrix operations. This project utilizes the reversible embedded domain-specific language NiLang.jl to differentiate sparse matrix operations by writing the sparse matrix operations in a reversible style. The generated backward rules are ported to ChainRules.jl as an extension, so that one can access these features in an automatic differentiation package like Zygote, Flux and Diffractor directly.
\subsection{Time Planning}

This project is shipped by four (mostly) sequential stages:
\begin{enumerate}[(1)]
    \item Implement low level operations by NiLang.
    \item Carefully test by CI and export chain rules into ChainRules.jl. 
    \item Implement high level operations and perform AD operations.
    \item Add some use cases and enhance docs. 
\end{enumerate}

I list the timeline of this project in the form of Gantt chart.
\begin{enumerate}[(1)]
    \item 1st Jul - 31st Jul \quad Implement low level operations by NiLang. 
    \item 1st Aug - 15th Aug   \quad Carefully test by CI and export chain rules into ChainRules.jl.
    \item 15th Sep - 20th Sep \quad  Implement high level operations and perform AD operations. 
    \item 21st Sep - 30th Sep \quad Add some use cases and enhance docs.
\end{enumerate}
\vspace{0.5cm}

\noindent\resizebox{\textwidth}{!}{
\begin{tikzpicture}[x=.5cm, y=1cm] 
\centering
        \begin{ganttchart}[%Specs
            y unit title=0.4cm,
            y unit chart=0.5cm,
            canvas/.style={fill=none, draw=black, line width=.75pt},
            vgrid,
            title label anchor/.style={below=-1.6ex},
            title left shift=.05,
            title right shift=-.05,
            title height=1,
            title/.style={fill=none},
            title label font=\bfseries,
            bar/.style={fill=barblue},
            incomplete/.style={fill=white},
            progress label text={},
            bar height=0.7,
            group right shift=0,
            group top shift=.6,
            group height=.3,
            group peaks height=.2]{1}{36}

            %labels
            \gantttitle{2021}{36}\\
            \gantttitle{Jul}{12}
            \gantttitle{Aug}{12}
            \gantttitle{Sep}{12}
            \\

            % Parameter Selection
            \ganttgroup{Stage 1}{2}{12}\\ %elem0
            %\ganttbar[progress=0]{sub-objective}{2}{18}\\
            \ganttmilestone{Low-Level Ops}{6}\\\\

            % Testing equipment
            \ganttgroup{Stage 2}{13}{25}\\ %elem0
            %\ganttbar[progress=0]{Bar}{2}{18}\\
            \ganttmilestone{CI and ChainRules}{18}\\\\

            % Testing
            \ganttgroup{Stage 3}{26}{32}\\ 
            %\ganttbar[progress=0]{Bar}{19}{35}\\
            %\ganttbar[progress=0]{Bar}{19}{35}\\
            \ganttmilestone{High-Level Ops}{29}\\
            %\ganttmilestone{Milestone}{35}\\

            % Algorithm Development
            \ganttgroup{Stage 4}{33}{35}\\
            \ganttmilestone{Merge}{34}\\ 
            %\ganttbar[progress=0]{Bar}{28}{35} \\

            %relations 
            %\ganttlink{elem2}{elem6}
            %\ganttlink{elem5}{elem6}
            %\ganttlink{elem9}{elem11}

        \end{ganttchart}
    \label{fig:Gantt2014}
\end{tikzpicture}}  

\vspace{0.5cm}  

\section{Project Summary}
\subsection{Project Output}
\begin{enumerate}[(1)]
    \item Differentiate sparse matrix operations in Julia base by rewriting the sparse functions in NiLang.jl.  
    \textbf{Status: } \textit{Already completed. Black-box tests on NiLang programs have passed in CI.}  
    \href{https://github.com/jieli-matrix/SparseArraysAD.jl/pull/20}{Related PR}
    \item Port the generated backward rules to ChainRules.jl as an extension. \\ 
    \textbf{Status: } \textit{Already completed. Backward rules have been checked by ChainRulesTestUtils.}  
    \href{https://github.com/jieli-matrix/SparseArraysAD.jl/pull/39}{Related PR}
    \item Release an open source julia package with test coverage over 85\%.  \\
    \textbf{Status: } \textit{Test coverage achieved at 87\%.} 
    \href{https://app.codecov.io/gh/jieli-matrix/SparseArraysAD.jl}{Code Converage Report} 
    \item Add some use cases for getting a start.  \\
    \textbf{Status: } \textit{Already completed. A simple use case is shown in ReadMe.}  
    \href{https://github.com/jieli-matrix/SparseArraysAD.jl/pull/34}{Related PR}
\end{enumerate}
\subsection{Scheme Progress}
\begin{enumerate}
    \item 1st Jul - 31st Jul \quad Implement low level operations by NiLang.  \\
    \textit{I have rewritten almost all sparse matrix multiplication and dot operations in Julia base by NiLang.  
    I list the implemented sparse matrix operators as follows. 
    All the functions listed have passed CI tests and the code coverage have achieved over 85\%.}
    \begin{lstlisting}[language=Julia]
        function imul!(C::StridedVecOrMat, A::AbstractSparseMatrix{T}, B::DenseInputVecOrMat, α::Number, β::Number) where T
        function imul!(C::StridedVecOrMat, xA::Adjoint{T, <:AbstractSparseMatrix}, B::DenseInputVecOrMat, α::Number, β::Number) where T
        function imul!(C::StridedVecOrMat, X::DenseMatrixUnion, A::AbstractSparseMatrix{T}, α::Number, β::Number) where T
        function imul!(C::StridedVecOrMat, X::Adjoint{T1, <:DenseMatrixUnion}, A::AbstractSparseMatrix{T2}, α::Number, β::Number) where {T1, T2}
        function imul!(C::StridedVecOrMat, X::DenseMatrixUnion, xA::Adjoint{T, <:AbstractSparseMatrix}, α::Number, β::Number) where T
        function idot(r, A::SparseMatrixCSC{T},B::SparseMatrixCSC{T}) where {T}
        function idot(r, x::AbstractVector, A::AbstractSparseMatrix{T1}, y::AbstractVector{T2}) where {T1, T2}
        function idot(r, x::SparseVector, A::AbstractSparseMatrix{T1}, y::SparseVector{T2}) where {T1, T2}
    \end{lstlisting}  
    \item 1st Aug - 15th Aug   \quad Carefully test by CI and export chain rules into ChainRules.jl.  \\
    \textit{I have exported the backward rules into ChainRules.jl. I used ChainRulesTestUtils.jl to test correctness and robustness of 
    sparse matrix AD rules. All the rules have passed CI tests.}
    \item 15th Sep - 20th Sep \quad  Implement high level operations and perform AD operation. \\
    \textit{I have implemented low rank svd algorithm in Julia. To perform AD on low\_rank\_svd algorithm, I wrapped backward rules of 
    QR decomposition since QR rules haven't been ensembled in ChainRules. I used finite differences method to check the correctness 
    of the algorithm. All the code have passed CI tests and the code converage have achieved at 87\% finally.}
    \item 21st Sep - 30th Sep \quad Add some use cases and enhance docs. \\
    \textit{A simple use case has been added into ReadMe in the project.}
\end{enumerate}
    
\subsection{Problems and Solutions}  
You could focus on the summary and experience
\subsection{Development Quality}
You could give a self evaluation of development quality
\subsection{Communication and Feedback with Mentor}

